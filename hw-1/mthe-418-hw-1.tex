% region Filename parsing.
% Provides macros manipulating strings of tokens.
\RequirePackage{xstring}

% Store the jobname as a string with category 11 characters.
\edef\normaljobname{\expandafter\scantokens\expandafter{\jobname\noexpand}}
\StrBehind{\normaljobname}{hw-}[\homeworknumber]
% endregion

\documentclass[
  coursecode={MTHE 418},
  assignmentname={Homework \homeworknumber},
  studentnumber=20053722,
  name={Bryan Hoang},
  draft,
  % final,
]{
  ltxanswer%
}

\usepackage{bch-style}
\usepackage{fontspec}
\usepackage{lstfiracode}

\setmonofont{FiraCode}[
  Contextuals=Alternate,
]

\ActivateVerbatimLigatures{}

\begin{document}
  \begin{questions}
    \question\
    \begin{parts}
      \part{}
      \begin{solution}
        Using Table 1.11, the ciphertext of the plaintext message is
        \begin{equation*}
          \text{\texttt{IBXFEPAQLBQAAXWQWIBXFSVAXW}}
        \end{equation*}
      \end{solution}

      \part{}
      \begin{solution}
        % \setlength\tabcolsep{3pt}
        \begin{table}
          \ttfamily
          \caption{The associated decryption table of Table 1.11.}
          \label{table:decryption-table}
          \begin{tblr}{
              XXXXXXXXXXXXXXXXXXXXXXXXXX
            }
            \toprule
            d & h & b & w & o & g & u & q & t & c & j & s & y & x & z & l & i & m & a & k & f & r & n & e & v & p \\
            \midrule
            A & B & C & D & E & F & G & H & I & J & K & L & M & N & O & P & Q & R & S & T & U & V & W & X & Y & Z \\
            \bottomrule
          \end{tblr}
        \end{table}
        \setlength\tabcolsep{6pt}
      \end{solution}

      \part{}
      \begin{solution}
        Using~\autoref{table:decryption-table} to decrypt the message yields the following plaintext message:
        \begin{equation*}
          \text{\texttt{The secret password is sword fish.}}
        \end{equation*}
      \end{solution}
    \end{parts}

    \question\
    \begin{parts}
      \part{}
      \begin{solution}
        \begin{proof}
          Let \(g = \gcd(a,b)\). Then \(\exists A, B \in \Z\) such that \(a = gA\) and \(b = gB\). Then substituting the equations into the given one yields
          \begin{align*}
            1 &= au + bv    \\
              &= gAu + gBv  \\
              &= g(Au + Bv) \\
          \end{align*}
          where \(Au + Bv \in Z\). Therefore, \(g\) divides 1, implying that \(g = 1\).
        \end{proof}
      \end{solution}

      \part{}
      \begin{solution}
        It is not necessarily true that \(\gcd(a,b) = 6\). For example, take \(a = 1\) and \(b = 2\). Then
        \begin{equation*}
          a \cdot (-6) + b \cdot 6 = 6,
        \end{equation*}
        and yet \(\gcd(a,b) = 1\).

        \begin{claim}
          In general, all possible values of \(\gcd(a,b)\) divide 6, i.e., the RHS of \(au + bv = 6\).
        \end{claim}
        \begin{proof}
          Suppose that \(au + bv = c\) has a solution. Let \(g = \gcd(a,b)\) and divide \(c\) by \(g\) with remainder to get
          \begin{equation*}
            c = gq + r, \quad \text{with}\ q,r \in \Z,\ 0 \le r < g.
          \end{equation*}
          Then by the extended euclidean algorithm, we can find \(x,y \in \Z\) such that \(g = ax + by\). Then
          \begin{gather*}
            au + bv = c = gq + r = (ax + by)q + r \\
            \Rightarrow a(u - xq) + b(v - yq) = r.
          \end{gather*}
          \(g\) divides the LHS since \(g\) divides both \(a\) and \(b\), which implies that \(g \mid r\). But if \(0 \le r < g\) and \(g \mid r\), then we have that \(r = 0\). Therefore, \(c = gq\) which means that g divides c, where \(c = 6\) for the specific example.
        \end{proof}
      \end{solution}

      \part{}
      \begin{solution}

      \end{solution}

      \part{}
      \begin{solution}
        \begin{proof}
          Let's subtract one equation from the other to get
          \begin{align*}
            au + bv - au_{0} - bv_{0} &= 0                                                      \\
            a(u - u_{0})              &= -b(v - v_{0})\numberthis\label{eq:subtract-equations}.
          \end{align*}
          Dividing both sides by \(g\) yields
          \begin{equation}\label{eq:divide-by-g}
            \frac{a}{g}(u - u_{0}) = -\frac{b}{g}(v - v_{0})
          \end{equation}
          We also have that
          \begin{align*}
            au + bv                                 &= g \\
            \Rightarrow \frac{a}{g}u + \frac{b}{g}v &= 1
          \end{align*}
          which, combined with part~(\ref{part@2@1}), gives \(\gcd(\frac{a}{g}, \frac{b}{g}) = 1\). By~\eqref{}
        \end{proof}
      \end{solution}
    \end{parts}

    \question{}
    \begin{solution}

    \end{solution}

    \question{}
    \begin{solution}

    \end{solution}

    \question{}
    \begin{solution}

    \end{solution}

    \question{}
    \begin{solution}

    \end{solution}

    \question{}
    \begin{solution}

    \end{solution}

    \question{}
    \begin{solution}

    \end{solution}

    \question{}
    \begin{solution}

    \end{solution}

    \question{}
    \begin{solution}

    \end{solution}

    \question{}
    \begin{solution}

    \end{solution}

    \question{}
    \begin{solution}

    \end{solution}

    \question{}
    \begin{solution}

    \end{solution}

    \question{}
    \begin{solution}

    \end{solution}

    \question{}
    \begin{solution}

    \end{solution}
  \end{questions}
\end{document}
