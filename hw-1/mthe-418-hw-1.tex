% region Filename parsing.
% Provides macros manipulating strings of tokens.
\RequirePackage{xstring}

% Store the jobname as a string with category 11 characters.
\edef\normaljobname{\expandafter\scantokens\expandafter{\jobname\noexpand}}
\StrBehind{\normaljobname}{hw-}[\homeworknumber]
% endregion

\documentclass[
  coursecode={MTHE 418},
  assignmentname={Homework \homeworknumber},
  studentnumber=20053722,
  name={Bryan Hoang},
  draft,
  final,
]{
  ltxanswer%
}

\usepackage{bch-style}
\usepackage{fontspec}
\usepackage{lstfiracode}

\setmonofont{FiraCode}[
  Contextuals=Alternate,
]

\ActivateVerbatimLigatures{}

\begin{document}
  \begin{questions}
    \question\
    \begin{parts}
      \part{}
      \begin{solution}
        Using Table 1.11, the ciphertext of the plaintext message is
        \begin{equation*}
          \text{\texttt{IBXFEPAQLBQAAXWQWIBXFSVAXW}}
        \end{equation*}
      \end{solution}

      \part{}
      \begin{solution}
        % \setlength\tabcolsep{3pt}
        \begin{table}
          \ttfamily
          \caption{The associated decryption table of Table 1.11.}
          \label{table:decryption-table}
          \begin{tblr}{
              XXXXXXXXXXXXXXXXXXXXXXXXXX
            }
            \toprule
            d & h & b & w & o & g & u & q & t & c & j & s & y & x & z & l & i & m & a & k & f & r & n & e & v & p \\
            \midrule
            A & B & C & D & E & F & G & H & I & J & K & L & M & N & O & P & Q & R & S & T & U & V & W & X & Y & Z \\
            \bottomrule
          \end{tblr}
        \end{table}
        \setlength\tabcolsep{6pt}
      \end{solution}

      \part{}
      \begin{solution}
        Using~\autoref{table:decryption-table} to decrypt the message yields the following plaintext message:
        \begin{equation*}
          \text{\texttt{The secret password is sword fish.}}
        \end{equation*}
      \end{solution}
    \end{parts}

    \question\
    \begin{parts}
      \part{}
      \begin{solution}
        \begin{proof}
          Let \(g = \gcd(a,b)\). Then \(\exists A, B \in \Z\) such that \(a = gA\) and \(b = gB\). Then substituting the equations into the given one yields
          \begin{align*}
            1 &= au + bv    \\
              &= gAu + gBv  \\
              &= g(Au + Bv) \\
          \end{align*}
          where \(Au + Bv \in Z\). Therefore, \(g\) divides 1, implying that \(g = 1\).
        \end{proof}
      \end{solution}

      \part{}
      \begin{solution}
        It is not necessarily true that \(\gcd(a,b) = 6\). For example, take \(a = 1\) and \(b = 2\). Then
        \begin{equation*}
          a \cdot (-6) + b \cdot 6 = 6,
        \end{equation*}
        and yet \(\gcd(a,b) = 1\).

        \begin{claim}
          In general, all possible values of \(\gcd(a,b)\) divide 6, i.e., the RHS of \(au + bv = 6\).
        \end{claim}
        \begin{proof}
          Suppose that \(au + bv = c\) has a solution. Let \(g = \gcd(a,b)\) and divide \(c\) by \(g\) with remainder to get
          \begin{equation*}
            c = gq + r, \quad \text{with}\ q,r \in \Z,\ 0 \le r < g.
          \end{equation*}
          Then by the extended euclidean algorithm, we can find \(x,y \in \Z\) such that \(g = ax + by\). Then
          \begin{gather*}
            au + bv = c = gq + r = (ax + by)q + r \\
            \Rightarrow a(u - xq) + b(v - yq) = r.
          \end{gather*}
          \(g\) divides the LHS since \(g\) divides both \(a\) and \(b\), which implies that \(g \mid r\). But if \(0 \le r < g\) and \(g \mid r\), then we have that \(r = 0\). Therefore, \(c = gq\) which means that g divides c, where \(c = 6\) for the specific example.
        \end{proof}
      \end{solution}

      \part{}
      \begin{solution}

      \end{solution}

      \part{}
      \begin{solution}
        \begin{proof}
          Let's subtract one equation from the other to get
          \begin{align*}
            au + bv - au_{0} - bv_{0} &= 0              \\
            a(u - u_{0})              &= -b(v - v_{0}).
          \end{align*}
          Dividing both sides by \(g\) yields
          \begin{equation}\label{eq:divide-by-g}
            \frac{a}{g}(u - u_{0}) = -\frac{b}{g}(v - v_{0})
          \end{equation}
          We also have that
          \begin{align*}
            au + bv                                 &= g \\
            \Rightarrow \frac{a}{g}u + \frac{b}{g}v &= 1
          \end{align*}
          which, combined with part~(\ref{part@2@1}), gives \(\gcd(\frac{a}{g}, \frac{b}{g}) = 1\). By~\eqref{eq:divide-by-g}, \(\frac{b}{g} \mid \frac{a}{g}(u - u_{0})\). Since \(\frac{b}{g}\) is relatively prime to \(\frac{a}{g}\), it follows that \(\frac{b}{g} \mid (u - u_{0})\). Thus
          \begin{equation*}
            u - u_{0} = \frac{b}{g}x \quad \text{for some}\ x \in \Z.
          \end{equation*}
          Along the same lines of reasoning, we can also say that
          \begin{equation*}
            v - v_{0} = \frac{a}{g}y \quad \text{for some}\ y \in \Z.
          \end{equation*}
          Therefore,
          \begin{equation*}
            u = u_{0} + \frac{b}{g}x \quad \text{and}\ v = v_{0} + \frac{a}{g}y.
          \end{equation*}
          Substituting it into~\eqref{eq:divide-by-g} gives
          \begin{align*}
            \frac{a}{g}\frac{b}{g}x &= -\frac{b}{g}\frac{a}{g}y \\
            \Rightarrow x           &= -y.
          \end{align*}
          If we let \(k = x\), then we have
          \begin{equation*}
            u = u_{0} + \frac{b}{g}k \quad \text{and}\ v = v_{0} + \frac{a}{g}k.
          \end{equation*}
        \end{proof}
      \end{solution}
    \end{parts}

    \question\
    \begin{parts}
      \part{}
      \begin{solution}
        \begin{equation*}
          x \equiv 23 - 17 \equiv \boxed{6} \Mod{n}.
        \end{equation*}
      \end{solution}

      \addtocounter{partno}{1}

      \part{}
      \begin{solution}
        The squares modulo 11 are \(0^{2} \equiv 0, 1^{2} \equiv 1, 2^{2} \equiv 4, 3^{2} \equiv 9, 4^{2} \equiv 5, 5^{2} \equiv 3, 6^{2} \equiv 3, 7^{2} \equiv 5, 8^{2} \equiv 9, 9^{2} \equiv 4, \text{ and } 10^{2} \equiv 1\). Then with \(5^{2} \equiv 3\) and \(6^{2} \equiv 3\), the two solutions are \fbox{\(x = 5\) and \(x = 6\)}.
      \end{solution}

      \addtocounter{partno}{2}

      \part{}
      \begin{solution}
        By substituting in \(x = 0, 1, 2, \cdots, 10\) into \(x^{3} - x^{2} + 2x - 2\) and reducing modulo 11, the three values that satisfy the equation are \fbox{\(x = 1\), \(x = 3\), and \(x = 8\)}.
      \end{solution}

      \part{}
      \begin{solution}
        The solutions to \(x \equiv 2 \mod{7}\) satisfying \(0 \le x \le 34\) are 2, 9, 16, 23, and 30. Reducing them modulo 5 respectively gives 2, 4, \underline{1}, 3, and 0. Therefore, the solution is \fbox{\(x = 16\)}.
      \end{solution}
    \end{parts}

    \question{}
    \begin{solution}
      \begin{proof}
        \begin{proofpart} First, let's assume that \(m\) is prime.

          Let \(a \in \Z\) be such that \(1 \le a < m \) and let \(g = \gcd(a,m)\). Then \(g \mid m\), which, combined with the fact that \(m\) is prime, implies that either \(g = 1\) or \(g = m\). But \(g \mid a\) and \(1 \le a < m\) as well, which implies that \(a = 1\).

          Then \(\forall a \in \Z\) such that \(1 \le a < m\), we have that \(\gcd(a,m)=1\). Therefore,
          \begin{align*}
            \phi(m) &= \#\{1 \le a < m:\gcd(a,m)=1\} \\
                    &= \#\{1, 2, \cdots, m-1\}       \\
                    &= m-1
          \end{align*}
        \end{proofpart}
        \begin{proofpart}
          Now assume that \(\phi(m) = m-1\).

          Then \(\forall a \in \Z\) such that \(1 \le a < m\), we have that \(\gcd(a,m)=1\). Suppose that \(a \mid m\) and that \(a \ne m\). Then \(1 \le a < m\), so \(\gcd(a,m)=1\). But \(a \mid m \Rightarrow \gcd(a,m)=a\). Therefore \(a = 1\). Since the only dividors of \(m\) are 1 and \(m\), we have that \(m\) is prime.
        \end{proofpart}
      \end{proof}
    \end{solution}

    \question\
    \begin{parts}
      \part{}
      \begin{solution}
        \begin{equation*}
          \boxed{x = 31}
        \end{equation*}
      \end{solution}

      \part{}
      \begin{solution}
        \begin{equation*}
          \boxed{x = 5764}
        \end{equation*}
      \end{solution}

      \part{}
      \begin{solution}
        \begin{equation*}
          \boxed{x = 221}
        \end{equation*}
      \end{solution}

      \part{}
      Note that the proposition to prove is a case of the Chinese remainder theorem.
      \begin{solution}
        \begin{proof}
          Assume that \(\gcd(m,n) = 1\). Then for any \(y \in \Z\), the solutions to the first congruence are of the form \(x =  + my\). Substituting in the second congruence gives
          \begin{equation*}
            a + my \equiv b \Mod{n},
          \end{equation*}
          which implies that, we need to find \(z \in Z\) such that
          \begin{align*}
            a + my - b          &= nz     \\
            \Rightarrow my - nz &= b - a.
          \end{align*}
          By assumption, \(\gcd(m,n) = 1\), so \(\exists u,v \in \Z\) satisfying
          \begin{align*}
            mu + nv                       &= 1      \\
            \Rightarrow mu(b-a) + nv(b-a) &= b - a.
          \end{align*}
          Now we can set \(y = u(b-a)\) and \(z = v(b-a)\). Thus,
          \begin{equation*}
            x = a + mu(b-a) = a + (1 - nv)(b - a) = b + nv(b - a),
          \end{equation*}
          which shows that \(x \equiv u \Mod{m}\) and \(x \equiv v \Mod{n}\).
        \end{proof}
      \end{solution}
    \end{parts}

    \question{}
    \begin{solution}

    \end{solution}

    \question{}
    \begin{solution}

    \end{solution}

    \question{}
    \begin{solution}

    \end{solution}

    \question{}
    \begin{solution}

    \end{solution}

    \question{}
    \begin{solution}

    \end{solution}

    \question{}
    \begin{solution}

    \end{solution}

    \question{}
    \begin{solution}

    \end{solution}

    \question{}
    \begin{solution}

    \end{solution}

    \question{}
    \begin{solution}

    \end{solution}

    \question{}
    \begin{solution}

    \end{solution}
  \end{questions}
\end{document}
