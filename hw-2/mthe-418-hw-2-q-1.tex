% region Filename parsing.
% Provides macros manipulating strings of tokens.
\RequirePackage{xstring}

% Store the jobname as a string with category 11 characters.
\edef\normaljobname{\expandafter\scantokens\expandafter{\jobname\noexpand}}
\StrBetween{\normaljobname}{hw-}{-q}[\homeworknumber]
\StrBehind{\normaljobname}{-q-}[\questionnumber]
% endregion

\documentclass[
  coursecode={MTHE 418},
  assignmentname={Homework \homeworknumber},
  studentnumber=20053722,
  name={Bryan Hoang},
  draft,
  % final,
]{
  ltxanswer%
}

\usepackage{bch-style}

\begin{document}
  \begin{questions}
    \setcounter{question}{\questionnumber}
    \addtocounter{question}{-1}
    \question[10]\
    \begin{parts}
      \part\
      \begin{subparts}
        \subpart{}
        \begin{solution}
          \begin{align*}
            e_{k}(m)     &\equiv k_{1} \cdot m + k_{2} \Mod{p}                                                                                                                        \\
                         &\equiv \begin{psmallmatrix}1&3\\2&2\end{psmallmatrix} \cdot \begin{psmallmatrix}2\\1\end{psmallmatrix} + \begin{psmallmatrix}5\\4\end{psmallmatrix} \Mod{7} \\
            \alignedbox{ &\equiv \begin{psmallmatrix}5\\3\end{psmallmatrix}} \Mod{7} .
          \end{align*}
        \end{solution}

        \subpart{}
        \begin{solution}
          The matrix \(k_{1}^{-1}\) used for decryption is \(\boxed{k_{1}^{-1}=\begin{psmallmatrix}3&6\\4&5\end{psmallmatrix}}\).
        \end{solution}

        \subpart{}
        \begin{solution}
          \begin{align*}
            d_{k}(c)     &\equiv k_{1}^{-1} \cdot (c - k_{2}) \Mod{p}                                                                                                                             \\
                         &\equiv \begin{psmallmatrix}3&6\\4&5\end{psmallmatrix} \cdot \bigl(\begin{psmallmatrix}3\\5\end{psmallmatrix} - \begin{psmallmatrix}5\\4\end{psmallmatrix}\bigr) \Mod{7} \\
            \alignedbox{ &\equiv \begin{psmallmatrix}0\\4\end{psmallmatrix}} \Mod{7}.
          \end{align*}
        \end{solution}
      \end{subparts}

      \part{}
      \begin{solution}
        The Hill cipher is vulnerable to a plaintext attack because each known plaintext and cipher text pair gives a congruence of the form \(c\equiv k_{1}\cdot m+k_{2}\). This yields \(n\) linear equations for the \(n^{2}+n=n\cdot(n+1)\) unknown entries of the keys \(k_{1}\) and \(k_{2}\). Thus, knowing \(n+1\) plaintext and ciphertext pairs for an attack would give enough equations for an attacke to solve for the keys \(k_{1}\) and \(k_{2}\).
      \end{solution}

      \part{}
      \begin{solution}
      \end{solution}

      \part{}
      \begin{solution}
      \end{solution}
    \end{parts}
  \end{questions}
\end{document}
