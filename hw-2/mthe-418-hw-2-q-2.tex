% region Filename parsing.
% Provides macros manipulating strings of tokens.
\RequirePackage{xstring}

% Store the jobname as a string with category 11 characters.
\edef\normaljobname{\expandafter\scantokens\expandafter{\jobname\noexpand}}
\StrBetween{\normaljobname}{hw-}{-q}[\homeworknumber]
\StrBehind{\normaljobname}{-q-}[\questionnumber]
% endregion

\documentclass[
  coursecode={MTHE 418},
  assignmentname={Homework \homeworknumber},
  studentnumber=20053722,
  name={Bryan Hoang},
  draft,
  % final,
]{
  ltxanswer%
}

\usepackage{bch-style}

\begin{document}
  \begin{questions}
    \setcounter{question}{\questionnumber}
    \addtocounter{question}{-1}
    \question[10]\
    \begin{parts}
      \part{}
      \begin{solution}
        \begin{itemize}
          \item \fbox{Yes}, \(e\) is indeed an encryption function.
          \item It's associated decryption function \(d\) is \(\boxed{d_{k}(c)\equiv k-c \Mod{N}}\).
        \end{itemize}
      \end{solution}

      \part{}
      \begin{solution}
        \begin{itemize}
          \item \fbox{No}, \(e\) is not an encryption function since it is not injective.
          \item We can make it an encryption function by restricting the set of keys to \(\mathcal{K}\equiv(\Z/N\Z)^{*} \Mod{N}\). Then it will have an associated decrytion function of \(d_{k}(c)\equiv k^{-1}c \Mod{N}\)
        \end{itemize}
      \end{solution}

      \part{}
      \begin{solution}
        \begin{itemize}
          \item \fbox{No}, \(e\) is not an encryption function since it is not injective.
          \item We cannot make it an encryption function by restricting the set of keys.
        \end{itemize}
      \end{solution}
    \end{parts}
  \end{questions}
\end{document}
