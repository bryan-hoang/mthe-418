% region Filename parsing.
% Provides macros manipulating strings of tokens.
\RequirePackage{xstring}

% Store the jobname as a string with category 11 characters.
\edef\normaljobname{\expandafter\scantokens\expandafter{\jobname\noexpand}}
\StrBetween{\normaljobname}{hw-}{-q}[\homeworknumber]
\StrBehind{\normaljobname}{-q-}[\questionnumber]
% endregion

\documentclass[
  coursecode={MTHE 418},
  assignmentname={Homework \homeworknumber},
  studentnumber=20053722,
  name={Bryan Hoang},
  draft,
  final,
]{
  ltxanswer%
}

\usepackage{bch-style}

\begin{document}
  \begin{questions}
    \setcounter{question}{\questionnumber}
    \addtocounter{question}{-1}
    \question[10]\
    \begin{parts}
      \part{}
      \begin{solution}
        \begin{proof}
          \begin{align*}
                                &\begin{cases}
                                   x=a \\
                                   x=b
                                 \end{cases}                                  \\
            \Rightarrow g^{a}   &\equiv h \equiv g^{b} \Mod{p}                 \\
            \Rightarrow g^{a-b} &\equiv 1 \Mod{p}\numberthis\label{eq:a-power}
          \end{align*}
          But since \(g\) is a primitive root,
          \begin{align*}
            \ord(g)         &= p-1\numberthis\label{eq:a-order}                                                        \\
            \Rightarrow p-1 &\divides a-b                       & &\text{by~\eqref{eq:a-power} and~\eqref{eq:a-order}} \\
            \Rightarrow a   &\equiv b \Mod{p-1}.
          \end{align*}
        \end{proof}
        The proven result implies that \(\log_{g}(h)\) is well-defined up to adding or subtracting multiples of \(p-1\), showing that the map (2.1) on page 63 is indeed well-defined.
      \end{solution}

      \part{}
      \begin{solution}
        \begin{proof}
          Let \(h_{1}, h_{2} \in \mathbb{F}_{p}^{8}\). Starting with the LHS, we have
          \begin{align*}
            g^{\log_{g}(h_{1}h_{2})}         &\equiv h_{1} h_{2} \Mod{p}                               \\
                                             &\equiv g^{\log_{g}(h_{1})} g^{\log_{g}(h_{2})} \Mod{p}   \\
                                             &\equiv g^{\log_{g}(h_{1})+\log_{g}(h_{2})} \Mod{p}       \\
            \Rightarrow \log_{g}(h_{1}h_{2}) &\equiv g^{\log_{g}(h_{1}) + \log_{g}(h_{2})}  \Mod{p-1}.
          \end{align*}
        \end{proof}
      \end{solution}

      \part{}
      \begin{solution}
        \begin{proof}
          Let \(h\in\mathbb{F}_{p}^{8}\). Starting with the RHS, we have
          \begin{align*}
            g^{n\log_{g}(h)}             &= \Bigl(g^{\log_{g}(h)}\Bigr)^{n}  \\
                                         &\equiv h^{n} \Mod{p}               \\
                                         &\equiv g^{\log_{g}(h^{n})} \Mod{p} \\
            \Rightarrow     n\log_{g}(h) &\equiv \log_{g}(h^{n}) \Mod{p-1}.
          \end{align*}
        \end{proof}
      \end{solution}
    \end{parts}
  \end{questions}
\end{document}
