% region Filename parsing.
% Provides macros manipulating strings of tokens.
\RequirePackage{xstring}

% Store the jobname as a string with category 11 characters.
\edef\normaljobname{\expandafter\scantokens\expandafter{\jobname\noexpand}}
\StrBetween{\normaljobname}{hw-}{-q}[\homeworknumber]
\StrBehind{\normaljobname}{-q-}[\questionnumber]
% endregion

\documentclass[
  coursecode={MTHE 418},
  assignmentname={Homework \homeworknumber},
  studentnumber=20053722,
  name={Bryan Hoang},
  draft,
  % final,
]{
  ltxanswer%
}

\usepackage{bch-style}

\begin{document}
  \begin{questions}
    \setcounter{question}{\questionnumber}
    \addtocounter{question}{-1}
    \question[10]\
    \begin{parts}
      \part{}
      \begin{solution}
        \begin{proof}
          Suppose we have an algorithm that solves the Diffie-Hellman problem. Then the algorithm can use the given values of \(g\), \(g^{a}\), and \(g^{b}\) to compute \(g^{ab}\). We can then compare the value of \(g^{ab}\) with the value of \(C\) to check if they are equal, thus solving the Diffie-Hellman decision problem.
        \end{proof}
      \end{solution}

      \part{}
      \begin{solution}
        I think the Diffie-Hellman decision problem is hard because I can't foresee a way of solving it without first solving the Diffie-Hellman problem, which is a hard problem.
      \end{solution}
    \end{parts}
  \end{questions}
\end{document}
