% region Filename parsing.
% Provides macros manipulating strings of tokens.
\RequirePackage{xstring}

% Store the jobname as a string with category 11 characters.
\edef\normaljobname{\expandafter\scantokens\expandafter{\jobname\noexpand}}
\StrBetween{\normaljobname}{hw-}{-q}[\homeworknumber]
\StrBehind{\normaljobname}{-q-}[\questionnumber]
% endregion

\documentclass[
  coursecode={MTHE 418},
  assignmentname={Homework \homeworknumber},
  studentnumber=20053722,
  name={Bryan Hoang},
  draft,
  final,
]{
  ltxanswer%
}

\usepackage{bch-style}

\begin{document}
  \begin{questions}
    \setcounter{question}{\questionnumber}
    \addtocounter{question}{-1}
    \question[10]\
    \begin{parts}
      \part{}
      \begin{solution}
        \begin{equation*}
          \tau\sigma^{2} = (\tau\sigma)\sigma = (\sigma^{2}\tau)\sigma = \sigma^{2}(\tau\sigma) = \sigma^{2}(\sigma^{2}\tau) = \sigma^{4}\tau = (\sigma^{3})\sigma\tau = \boxed{\sigma\tau}.
        \end{equation*}
      \end{solution}

      \part{}
      \begin{solution}
        \begin{equation*}
          \tau(\sigma\tau) = (\tau\sigma)\tau = (\sigma^{2}\tau)\tau = \sigma^{2}\tau^{2} = \boxed{\sigma^{2}}.
        \end{equation*}
      \end{solution}

      \part{}
      \begin{solution}
        \begin{equation*}
          (\sigma\tau)(\sigma\tau) = \sigma(\tau\sigma)\tau = \sigma(\sigma^{2}\tau)\tau = \sigma^{3}\tau^{2} = \boxed{e}.
        \end{equation*}
      \end{solution}

      \part{}
      \begin{solution}
        \begin{equation*}
          (\sigma\tau)(\sigma^{2}\tau) = \sigma(\tau\sigma)\sigma\tau = \sigma(\sigma^{2}\tau)\sigma\tau = \sigma^{3}(\tau\sigma)\tau = e(\sigma^{2}\tau)\tau = \sigma^{2}\tau^{2} = \boxed{\sigma^{2}}.
        \end{equation*}
      \end{solution}
    \end{parts}

    \begin{solution}
      \(\mathcal{S}_{3}\) is not a commutative group. For instance, the third rule of multiplication says that \(\tau\sigma=\sigma^{2}\tau\ne\sigma\tau\).
    \end{solution}
  \end{questions}
\end{document}
