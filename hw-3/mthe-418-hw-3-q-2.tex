% region Filename parsing.
% Provides macros manipulating strings of tokens.
\RequirePackage{xstring}

% Store the jobname as a string with category 11 characters.
\edef\normaljobname{\expandafter\scantokens\expandafter{\jobname\noexpand}}
\StrBetween{\normaljobname}{hw-}{-q}[\homeworknumber]
\StrBehind{\normaljobname}{-q-}[\questionnumber]
% endregion

\documentclass[
  coursecode={MTHE 418},
  assignmentname={Homework \homeworknumber},
  studentnumber=20053722,
  name={Bryan Hoang},
  draft,
  final,
]{
  ltxanswer%
}

\usepackage{bch-style}

\date{2022-02-28}

\begin{document}
  \begin{questions}
    \setcounter{question}{\questionnumber}
    \addtocounter{question}{-1}
    \question[10]\
    \begin{parts}
      \setcounter{partno}{3}
      \part{}
      \begin{solution}
        \begin{align*}
          \lim_{k\to\infty} \frac{(\ln k)^{375}}{k^{0.001}} &= \lim_{k\to\infty} \frac{375 (\ln k)^{374} k^{-1}}{0.001 k^{-0.999}} & &\text{by L'Hopital's rule} \\
                                                            &= \lim_{k\to\infty} \frac{375 (\ln k)^{374}}{0.001 k^{0.001}}.                                      \\
          \intertext{Applying L'Hopital's rule over and over again yields}
          \lim_{k\to\infty} \frac{(\ln k)^{375}}{k^{0.001}} &= \lim_{k\to\infty} \frac{375!}{0.001^{375} k^{0.001}}                                              \\
                                                            &= 0                                                                                                 \\
                                                            &< \infty
        \end{align*}
        \(\therefore\ (\ln k)^{375} = \mathcal{O}(k^{0.001})\).
      \end{solution}

      \part{}
      \begin{solution}
        \begin{align*}
          \lim_{k\to\infty} \frac{k^{2} 2^{k}}{e^{2k}} &< \lim_{k\to\infty} \frac{k^{2} e^{k}}{e^{2k}} \\
                                                       &= \lim_{k\to\infty} \frac{k^{2}}{e^{k}}        \\
                                                       &= 0                                            \\
                                                       &< \infty.
        \end{align*}
        \(\therefore\ k^{2} 2^{k} = \mathcal{O}(e^{2k})\).
      \end{solution}

      \part{}
      \begin{solution}
        \begin{align*}
          \lim_{N\to\infty} \frac{N^{10}2^{N}}{e^{N}} &= \lim_{N\to\infty} \frac{N^{10}}{\bigl(\frac{1}{2}\bigr)^{N} e^{N}} \\
                                                      &=\lim_{N\to\infty} \frac{N^{10}}{\bigl(\frac{e}{2}\bigr)^{N}}        \\
                                                      &= 0                                                                  \\
                                                      &< \infty.
        \end{align*}
        \(\therefore\ N^{10} 2^{N} = \mathcal{O}(e^{N})\).
      \end{solution}
    \end{parts}
  \end{questions}
\end{document}
