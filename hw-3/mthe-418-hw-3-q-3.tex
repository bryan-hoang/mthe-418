% region Filename parsing.
% Provides macros manipulating strings of tokens.
\RequirePackage{xstring}

% Store the jobname as a string with category 11 characters.
\edef\normaljobname{\expandafter\scantokens\expandafter{\jobname\noexpand}}
\StrBetween{\normaljobname}{hw-}{-q}[\homeworknumber]
\StrBehind{\normaljobname}{-q-}[\questionnumber]
% endregion

\documentclass[
  coursecode={MTHE 418},
  assignmentname={Homework \homeworknumber},
  studentnumber=20053722,
  name={Bryan Hoang},
  draft,
  % final,
]{
  ltxanswer%
}

\usepackage{bch-style}

\date{2022-02-28}

\begin{document}
  \begin{questions}
    \setcounter{question}{\questionnumber}
    \addtocounter{question}{-1}
    \question[10]\
    \begin{parts}
      \part{}
      \begin{solution}
        To solve \(11^{x} = 21\) in \(\mathbb{F}_{71}\), we first note that 11 has order 70 in \(\mathbb{F}_{71}\). Let \(n = 1 + \lfloor 70 \rfloor = 9\). After writing code to implement the algorithm, we find that \(\boxed{x = 37}\).
      \end{solution}

      \part{}
      \begin{solution}
        To solve \(156^{x} = 116\) in \(\mathbb{F}_{593}\), we first note that 156 has order 148 in \(\mathbb{F}_{593}\). Let \(n = 1 + \lfloor 148 \rfloor = 13\). After writing code to implement the algorithm, we find that \(\boxed{x = 59}\).
      \end{solution}

      \part{}
      \begin{solution}
        To solve \(650^{x} = 2213\) in \(\mathbb{F}_{3571}\), we first note that 650 has order 510 in \(\mathbb{F}_{3571}\). Let \(n = 1 + \lfloor 510 \rfloor = 23\). After writing code to implement the algorithm, we find that \(\boxed{x = 319}\).
      \end{solution}
    \end{parts}
  \end{questions}
\end{document}
