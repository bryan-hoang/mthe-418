% region Filename parsing.
% Provides macros manipulating strings of tokens.
\RequirePackage{xstring}

% Store the jobname as a string with category 11 characters.
\edef\normaljobname{\expandafter\scantokens\expandafter{\jobname\noexpand}}
\StrBetween{\normaljobname}{hw-}{-q}[\homeworknumber]
\StrBehind{\normaljobname}{-q-}[\questionnumber]
% endregion

\documentclass[
  coursecode={MTHE 418},
  assignmentname={Homework \homeworknumber},
  studentnumber=20053722,
  name={Bryan Hoang},
  draft,
  % final,
]{
  ltxanswer%
}

\usepackage{bch-style}

\date{2022-02-28}

\begin{document}
  \begin{questions}
    \setcounter{question}{\questionnumber}
    \addtocounter{question}{-1}
    \question[10]\
    \begin{parts}
      \addtocounter{partno}{2}
      \part{}
      \begin{solution}
        \(1159 = 19 \cdot 61\) and \((19 - 1)\cdot (61 - 1) = 1080\). With \(e = 73\), the congruence \(73d \equiv 1 \Mod{1080}\) has the solution \(d \equiv 557 \Mod{1080}\). Therefore, \(\boxed{x \equiv 614^{577} \equiv 158 \Mod{1159}}\).
      \end{solution}

      \part{}
      \begin{solution}
        \(8023 = 71 \cdot 113\) and \((71 - 1) \cdot (113 - 1) = 7840\). With \(e = 751\), the congruence \(751d \equiv 1 \Mod{7840}\) has the solution \(d \equiv 7151 \Mod{7840}\). Therefore, \(\boxed{x \equiv 677^{7151} \equiv 1355 \Mod{8023}}\).
      \end{solution}
    \end{parts}
  \end{questions}
\end{document}
