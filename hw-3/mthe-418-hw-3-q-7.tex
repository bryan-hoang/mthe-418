% region Filename parsing.
% Provides macros manipulating strings of tokens.
\RequirePackage{xstring}

% Store the jobname as a string with category 11 characters.
\edef\normaljobname{\expandafter\scantokens\expandafter{\jobname\noexpand}}
\StrBetween{\normaljobname}{hw-}{-q}[\homeworknumber]
\StrBehind{\normaljobname}{-q-}[\questionnumber]
% endregion

\documentclass[
  coursecode={MTHE 418},
  assignmentname={Homework \homeworknumber},
  studentnumber=20053722,
  name={Bryan Hoang},
  draft,
  % final,
]{
  ltxanswer%
}

\usepackage{bch-style}

\date{2022-02-28}

\begin{document}
  \begin{questions}
    \setcounter{question}{\questionnumber}
    \addtocounter{question}{-1}
    \question[10]\
    \begin{parts}
      \part{}
      \begin{solution}
        With \(\gcd(c, N) = 1\), then the formula in
        Exercise 3.4(c) says that
        \begin{equation}
          c^{\phi(N)} \equiv 1 \Mod{N}.
        \end{equation}
        Taking both sides of the congruence to the power of \(\phi{N}\) yields
        \begin{align*}
          \bigl(x^{e}\bigr)^{\phi(N)} &\equiv c^{\phi(N)} \Mod{N} \\
          \bigl(x^{e}\bigr)^{\phi(N)} &\equiv 1 \Mod{N}.
        \end{align*}
        To have the LHS satisfy the formula in Exercise 3.4(c), let \(d \equiv e^{-1} \Mod{\phi{N}}\). Then it is sufficient to find \(x = c^{d}\).
      \end{solution}

      \part{}\
      \begin{subparts}
        \subpart{}
        \begin{solution}
          To solve \(x^{577} \equiv 60 \Mod{1463}\), we first note that \(N = 7 \cdot 11 \cdot 19\). By the formula in Exercise 3.5(d),
          \begin{align*}
            \phi(1463) &= 1463 \biggl(1 - \frac{1}{7}\biggr) \biggl(1 - \frac{1}{11}\biggr) \biggl(1 - \frac{1}{19}\biggr) \\
                       &= 1080.
          \end{align*}
          Then \(d \equiv 577^{-1} \equiv 73 \Mod{1080}\). Therefore,
          \begin{equation*}
            \boxed{x = 60^{73} \equiv 1390 \Mod{1463}}.
          \end{equation*}
        \end{solution}

        \subpart{}
        \begin{solution}
          To solve \(x^{959} \equiv 1583 \Mod{1625}\), we first note that \(N = 5^{3} \cdot 13\). By the formula in Exercise 3.5(d),
          \begin{align*}
            \phi(1625) &= 1625 \biggl(1 - \frac{1}{5}\biggr) \biggl(1 - \frac{1}{13}\biggr) \\
                       &= 1200.
          \end{align*}
          Then \(d \equiv 959^{-1} \equiv 239 \Mod{1200}\). Therefore,
          \begin{equation*}
            \boxed{x = 1583^{239} \equiv 147 \Mod{1625}}.
          \end{equation*}
        \end{solution}

        \subpart{}
        \begin{solution}
          To solve \(x^{133957} \equiv 224689 \Mod{2134440}\), we first note that \(N = 2^{3} \cdot 3^{2} \cdot 5 \cdot 7^{2} \cdot 11^{2}\). By the formula in Exercise 3.5(d),
          \begin{align*}
            \phi(2134440) &= 2134440 \biggl(1 - \frac{1}{2}\biggr) \biggl(1 - \frac{1}{3}\biggr) \biggl(1 - \frac{1}{5}\biggr) \biggl(1 - \frac{1}{7}\biggr) \biggl(1 - \frac{1}{11}\biggr) \\
                          &= 443520.
          \end{align*}
          Then \(d \equiv 133957^{-1} \equiv 326413 \Mod{443520}\). Therefore,
          \begin{equation*}
            \boxed{x = 224689^{326413} \equiv 1892929 \Mod{2134440}}.
          \end{equation*}
        \end{solution}
      \end{subparts}
    \end{parts}
  \end{questions}
\end{document}
