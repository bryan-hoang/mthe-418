% region Filename parsing.
% Provides macros manipulating strings of tokens.
\RequirePackage{xstring}

% Store the jobname as a string with category 11 characters.
\edef\normaljobname{\expandafter\scantokens\expandafter{\jobname\noexpand}}
\StrBetween{\normaljobname}{hw-}{-q}[\homeworknumber]
\StrBehind{\normaljobname}{-q-}[\questionnumber]
% endregion Filename parsing.

\documentclass[
  coursecode={MTHE 418},
  assignmentname={Homework \homeworknumber},
  studentnumber=20053722,
  name={Bryan Hoang},
  draft,
  % final,
]{
  ltxanswer%
}

\usepackage{bch-style}

\begin{document}
  \begin{questions}
    \setcounter{question}{\questionnumber}
    \addtocounter{question}{-1}
    \question[10]\
    \begin{parts}
      \part{}
      \begin{solution}
        To decrypt the first ciphertext block \(c = 1794677960\), we compute
        \begin{align*}
          \legendre{1794677960}{32411} &= \legendre{16068}{32411}                        \\
                                       &= \legendre{2}{32411}^{2} \legendre{4017}{32411} \\
                                       &= \legendre{4017}{32411}                         \\
                                       &= \legendre{32411}{4017}                         \\
                                       &= \legendre{275}{4017}                           \\
                                       &= \legendre{167}{275}                            \\
                                       &= -\legendre{275}{167}                           \\
                                       &= -\legendre{108}{167}                           \\
                                       &= -\legendre{2}{167}^{2} \legendre{27}{167}      \\
                                       &= \legendre{167}{27}                             \\
                                       &= \legendre{5}{27}                               \\
                                       &= \legendre{27}{5}                               \\
                                       &= \legendre{2}{5}                                \\
                                       &= -1,
        \end{align*}
        which gives the plaintext bit \(m = 1\).

        \newpage

        To decrypt the second ciphertext block \(c = 525734818\), we compute
        \begin{align*}
          \legendre{525734818}{32411} &= \legendre{28398}{32411}                                           \\
                                      &= \legendre{2}{32411} \legendre{3}{32411} \legendre{4733}{32411}    \\
                                      &= \legendre{32411}{3} \legendre{32411}{4733}                        \\
                                      &= \legendre{2}{3} \legendre{4013}{4733}                             \\
                                      &= -\legendre{4733}{4013}                                            \\
                                      &= -\legendre{720}{4013}                                             \\
                                      &= -\legendre{2}{4013}^{4} \legendre{2}{4013}^{2} \legendre{5}{4013} \\
                                      &= -\legendre{4013}{5}                                               \\
                                      &= -\legendre{3}{5}                                                  \\
                                      &= -\legendre{5}{3}                                                  \\
                                      &= -\legendre{2}{3}                                                  \\
                                      &= 1,
        \end{align*}
        which gives the plaintext bit \(m = 0\).

        \newpage

        To decrypt the third ciphertext block \(c = 420526487\), we compute
        \begin{align*}
          \legendre{420526487}{32411} &= \legendre{26173}{32411}                                       \\
                                      &= \legendre{7}{32411} \legendre{3739}{32411}                    \\
                                      &= \legendre{32411}{7} \legendre{32411}{3739}                    \\
                                      &= \legendre{1}{7} \legendre{2499}{3739}                         \\
                                      &= \legendre{3}{3739} \legendre{7}{3739}^{2} \legendre{17}{3739} \\
                                      &= -\legendre{3739}{3} \legendre{3739}{17}                       \\
                                      &= -\legendre{1}{3} \legendre{16}{17}                            \\
                                      &= -\legendre{2}{17}^{4}                                         \\
                                      &= -1,
        \end{align*}
        which gives the plaintext bit \(m = 1\).

        Therefore, Alice's plaintext message is \(\boxed{(1, 0, 1)}\).
      \end{solution}

      \part{}
      \begin{solution}
        The factorization of \(N\) is \(N = pq = 47 \cdot 67\).

        To decrypt the first ciphertext block \(c = 2322\), we compute
        \begin{align*}
          \legendre{2322}{47} &= \legendre{19}{47}     \\
                              &= -\legendre{47}{19}    \\
                              &= -\legendre{9}{19}     \\
                              &= -\legendre{3}{19}^{2} \\
                              &= -1
        \end{align*}
        which gives the plaintext bit \(m = 1\).

        To decrypt the second ciphertext block \(c = 719\), we compute
        \begin{align*}
          \legendre{719}{47} &= \legendre{14}{47}                 \\
                             &= \legendre{2}{47} \legendre{7}{47} \\
                             &= -\legendre{47}{7}                 \\
                             &= -\legendre{5}{7}                  \\
                             &= -\legendre{7}{5}                  \\
                             &= -\legendre{2}{5}                  \\
                             &= 1
        \end{align*}
        which gives the plaintext bit \(m = 0\).

        To decrypt the third ciphertext block \(c = 202\), we compute
        \begin{align*}
          \legendre{202}{47} &= \legendre{14}{47}                 \\
                             &= \legendre{2}{47} \legendre{7}{47} \\
                             &= -\legendre{47}{7}                 \\
                             &= -\legendre{5}{7}                  \\
                             &= -\legendre{7}{5}                  \\
                             &= -\legendre{2}{5}                  \\
                             &= 1
        \end{align*}
        which gives the plaintext bit \(m = 0\).

        Therefore, Alice's plaintext message is \(\boxed{(1, 0, 0)}\).
      \end{solution}

      \part{}
      \begin{solution}
        To encrypt the first message bit \(m = 1\) using \(r = 705130839\), we compute
        \begin{align*}
          c &\equiv ar^{2} \pmod{781044643}                        \\
            &\equiv 568980706 \cdot 705130839^{2} \pmod{781044643} \\
            &\equiv 517254876 \pmod{781044643}.
        \end{align*}
        To encrypt the second message bit \(m = 1\) using \(r = 631364468\), we compute
        \begin{align*}
          c &\equiv ar^{2} \pmod{781044643}                        \\
            &\equiv 568980706 \cdot 631364468^{2} \pmod{781044643} \\
            &\equiv 4308279 \pmod{781044643}
        \end{align*}

        To encrypt the third message bit \(m = 0\) using \(r = 67651321\), we compute
        \begin{align*}
          c &\equiv ar^{2} \pmod{781044643}                       \\
            &\equiv 568980706 \cdot 67651321^{2} \pmod{781044643} \\
            &\equiv 660699010 \pmod{781044643}
        \end{align*}
        Therefore, the ciphertext for \((1, 1, 0)\) is \(\boxed{(517254876, 4308279, 660699010)}\).
      \end{solution}
    \end{parts}
  \end{questions}
\end{document}
