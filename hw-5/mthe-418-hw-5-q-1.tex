% region Filename parsing.
% Provides macros manipulating strings of tokens.
\RequirePackage{xstring}

% Store the jobname as a string with category 11 characters.
\edef\normaljobname{\expandafter\scantokens\expandafter{\jobname\noexpand}}
\StrBetween{\normaljobname}{hw-}{-q}[\homeworknumber]
\StrBehind{\normaljobname}{-q-}[\questionnumber]
% endregion Filename parsing.

\documentclass[
  coursecode={MTHE 418},
  assignmentname={Homework \homeworknumber},
  studentnumber=20053722,
  name={Bryan Hoang},
  draft,
  % final,
]{
  ltxanswer%
}

\usepackage{bch-style}

\begin{document}
  \begin{questions}
    \setcounter{question}{\questionnumber}
    \addtocounter{question}{-1}
    \question[10]\
    \begin{parts}
      \part{}
      \begin{solution}
        Computing \(P \oplus Q\) yields
        \begin{align*}
          \lambda                            &= \frac{y_{2} - y_{1}}{x_{2} - x_{1}}          \\
                                             &= \frac{5 - 4}{2 - (-1)}                       \\
                                             &= \frac{1}{3},                                 \\
          x_{3}                              &= \lambda^{2} - x_{1} - x_{2}                  \\
                                             &= \frac{1}{3}^{2} - (-1) - 2                   \\
                                             &= -\frac{8}{9},                                \\
          y_{3}                              &= \lambda(x_{1} - x_{3}) - y_{1}               \\
                                             &= \frac{1}{3} \Bigl(-1 + \frac{8}{9}\Bigr) - 4 \\
                                             &= -\frac{109}{27},                             \\
          \Rightarrow \alignedbox{P \oplus Q &= \Bigl(-\frac{8}{9}, -\frac{109}{27}\Bigr)}.
        \end{align*}
        Computing \(P \ominus Q = P \oplus (-Q) = (-1, 4) \oplus (2, -5) \) yields
        \begin{align*}
          \lambda                             &= \frac{y_{2} - y_{1}}{x_{2} - x_{1}} \\
                                              &= \frac{-5 - 4}{2 - (-1)}             \\
                                              &= -3,                                 \\
          x_{3}                               &= \lambda^{2} - x_{1} - x_{2}         \\
                                              &= (-3)^{2} - (-1) - 2                 \\
                                              &= 8,                                  \\
          y_{3}                               &= \lambda(x_{1} - x_{3}) - y_{1}      \\
                                              &= (-3) (-1 - 8) - 4                   \\
                                              &= 23,                                 \\
          \Rightarrow \alignedbox{P \ominus Q &= (8, 23)}.
        \end{align*}
      \end{solution}

      \newpage

      \part{}
      \begin{solution}
        Computing \(2P = P \oplus P = (-1, 4) \oplus (-1, 4)\) yields
        \begin{align*}
          \lambda                    &= \frac{3 x_{1}^{2} + A}{2 y_{1}}                 \\
                                     &= \frac{3 (-1)^{2} + 0}{2 (4)}                    \\
                                     &= \frac{3}{8},                                    \\
          x_{3}                      &= \lambda^{2} - x_{1} - x_{2}                     \\
                                     &= \Bigl(\frac{3}{8}\Bigr)^{2} - 2 (-1)            \\
                                     &= \frac{137}{64},                                 \\
          y_{3}                      &= \lambda(x_{1} - x_{3}) - y_{1}                  \\
                                     &= \frac{3}{8} \Bigl(-1 - \frac{137}{64}\Bigr) - 4 \\
                                     &= -\frac{2651}{512},                              \\
          \Rightarrow \alignedbox{2P &= \Bigl(\frac{137}{64}, -\frac{2651}{512}\Bigr)}.
        \end{align*}
        Computing \(2Q = Q \oplus Q = (2, 5) \oplus (2, 5)\) yields
        \begin{align*}
          \lambda                    &= \frac{3 x_{1}^{2} + A}{2 y_{1}}               \\
                                     &= \frac{3 (2)^{2} + 0}{2 (5)}                   \\
                                     &= \frac{6}{5},                                  \\
          x_{3}                      &= \lambda^{2} - x_{1} - x_{2}                   \\
                                     &= \Bigl(\frac{6}{5}\Bigr)^{2} - 2 (2)           \\
                                     &= -\frac{64}{25},                               \\
          y_{3}                      &= \lambda(x_{1} - x_{3}) - y_{1}                \\
                                     &= \frac{6}{5} \Bigl(2 + \frac{64}{25}\Bigr) - 5 \\
                                     &= \frac{59}{125},                               \\
          \Rightarrow \alignedbox{2Q &= \Bigl(-\frac{64}{25}, \frac{59}{125}\Bigr)}.
        \end{align*}
      \end{solution}
    \end{parts}
  \end{questions}
\end{document}
