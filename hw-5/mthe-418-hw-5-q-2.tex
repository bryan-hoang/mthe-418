% region Filename parsing.
% Provides macros manipulating strings of tokens.
\RequirePackage{xstring}

% Store the jobname as a string with category 11 characters.
\edef\normaljobname{\expandafter\scantokens\expandafter{\jobname\noexpand}}
\StrBetween{\normaljobname}{hw-}{-q}[\homeworknumber]
\StrBehind{\normaljobname}{-q-}[\questionnumber]
% endregion Filename parsing.

\documentclass[
  coursecode={MTHE 418},
  assignmentname={Homework \homeworknumber},
  studentnumber=20053722,
  name={Bryan Hoang},
  draft,
  final,
]{
  ltxanswer%
}

\usepackage{bch-style}

\begin{document}
  \begin{questions}
    \setcounter{question}{\questionnumber}
    \addtocounter{question}{-1}
    \question[10]
    \begin{solution}
      \begin{proof}
        Multiplying out the right-hand side gives
        \begin{equation*}
          X^{3} + AX + B = X^{3} - (e_{1} + e_{2} + e_{3}) X^{2} + (e_{1} e_{2} + e_{1} e_{3} + e_{2} e_{3}) X - e_{1} e_{2} e_{3}.
        \end{equation*}
        Comparing the coefficients yields
        \begin{empheq}[left=\empheqlbrace]{align*}
          0 &= e_{1} + e_{2} + e_{3}\numberthis\label{eq:pp1-sum}                   \\
          A &= e_{1} e_{2} + e_{1} e_{3} + e_{2} e_{3}\numberthis\label{eq:pp1-a} \\
          B &= e_{1} e_{2} e_{3}.\numberthis\label{eq:pp1-b}
        \end{empheq}
        \begin{proofpart}
          (\(\Rightarrow\)) First, suppose that
          \begin{equation}\label{eq:discriminant}
            4 A^{3} + 27 B^{2} = 0
          \end{equation}
          Then substituting~\eqref{eq:pp1-a} and~\eqref{eq:pp1-b} into~\eqref{eq:discriminant} yields
          \begin{equation*}
            0 = (4 e_{2}^{3} + 12 e_{3} e_{2}^{2} + 12 e_{3}^{2} e_{2} + 4 e_{3}^{3}) e_{1}^{3} + (12 e_{3} e_{2}^{3} + 51 e_{3}^{2} e_{2}^{2} + 12 e_{3}^{3} e_{2}) e_{1}^{2} + (12 e_{3}^{2} e_{2}^{3} + 12 e_{3}^{3} e_{2}^{2}) e_{1} + 4 e_{3}^{3} e_{2}^{3}.
          \end{equation*}
          Next, substituting in \(e_{1} = - e_{2} - e_{3}\) from~\eqref{eq:pp1-sum} lets us obtain
          \begin{align*}
            0 &= -4 e_{2}^{6} - 12 e_{3} e_{2}^{5} + 3 e_{3}^{2} e_{2}^{4} + 26 e_{3}^{3} e_{2}^{3} + 3 e_{3}^{4} e_{2}^{2} - 12 e_{3}^{5} e_{2} - 4 e_{3}^{6} \\
              &= -(e_{2} - e_{3})^{2} (e_{2} + 2 e_{3})^{2} (e_{3} + 2 e_{2})^{2}                                                                              \\
              &= (e_{2} - e_{3})^{2} (e_{1} - e_{3})^{2} (e_{1} - e_{2})^{2},
          \end{align*}
          which implies that at least two of the \(e_{i}\) are the same.
        \end{proofpart}
        \begin{proofpart}
          (\(\Leftarrow\)) Next, suppose that two of the \(e_{i}\) are the same, say \(e_{2} = e_{3}\). Then
          \begin{empheq}[left=\empheqlbrace]{align*}
            0 &= e_{1} + 2 e_{2}\numberthis\label{eq:pp2-sum} \\
            A &= 2 e_{1} e_{2} + e_{2}^{2}\numberthis\label{eq:pp2-a} \\
            B &= e_{1} e_{2}^{2}\numberthis\label{eq:pp2-b}.
          \end{empheq}
          Substituting~\eqref{eq:pp2-sum} into~\eqref{eq:pp2-a} and~\eqref{eq:pp2-b} gives
          \begin{empheq}[left=\empheqlbrace]{align*}
            A &= -3 e_{2}^{2} \\
            B &= -2 e_{2}^{3}.
          \end{empheq}
          Therefore,
          \begin{align*}
            4 A^{3} + 27 B^{2} &= 4 (-3 e_{2}^{2})^{3} + 27 (-2 e_{2}^{3})^{2} \\
                               &= 0.
          \end{align*}
        \end{proofpart}
      \end{proof}
    \end{solution}
  \end{questions}
\end{document}
