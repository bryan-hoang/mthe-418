% region Filename parsing.
% Provides macros manipulating strings of tokens.
\RequirePackage{xstring}

% Store the jobname as a string with category 11 characters.
\edef\normaljobname{\expandafter\scantokens\expandafter{\jobname\noexpand}}
\StrBetween{\normaljobname}{hw-}{-q}[\homeworknumber]
\StrBehind{\normaljobname}{-q-}[\questionnumber]
% endregion Filename parsing.

\documentclass[
  coursecode={MTHE 418},
  assignmentname={Homework \homeworknumber},
  studentnumber=20053722,
  name={Bryan Hoang},
  draft,
  % final,
]{
  ltxanswer%
}

\usepackage{bch-style}

\begin{document}
  \begin{questions}
    \setcounter{question}{\questionnumber}
    \addtocounter{question}{-1}
    \question[10]\
    \begin{parts}
      \part{}
      \begin{solution}
        \begin{table}
          \caption{Addition table for \(E(\mathbb{F}_{5}) = \{\mathcal{O}, (1, 2), (1, 3), (4, 0)\}\).}
          \renewcommand\arraystretch{1.3}
          \setlength\doublerulesep{0pt}
          \begin{tabular}{|c||*{4}{c|}}
            \hline
            +               & \(\mathcal{O}\) & (1, 2)          & (1, 3)          & (4, 0)          \\
            \hline
            \(\mathcal{O}\) & \(\mathcal{O}\) & (1, 2)          & (1, 3)          & (4, 0)          \\
            \hline
            (1, 2)          & (1, 2)          & (4, 0)          & \(\mathcal{O}\) & (1, 3)          \\
            \hline
            (1, 3)          & (1, 3)          & \(\mathcal{O}\) & (4, 0)          & (1, 2)          \\
            \hline
            (4, 0)          & (4, 0)          & (1, 3)          & (1, 2)          & \(\mathcal{O}\) \\
            \hline
          \end{tabular}
        \end{table}
      \end{solution}

      \part{}
      \begin{solution}
        \begin{table}
          \caption{Addition table for \(E(\mathbb{F}_{7}) = \{\mathcal{O}, (2, 1), (2, 6), (3, 1), (3, 6), (6, 0)\}\).}
          \renewcommand\arraystretch{1.3}
          \setlength\doublerulesep{0pt}
          \begin{tabular}{|c||*{6}{c|}}
            \hline
            +               & \(\mathcal{O}\) & (2, 1)          & (2, 6)          & (3, 1)          & (3, 6)          & (6, 0)          \\
            \hline
            \(\mathcal{O}\) & \(\mathcal{O}\) & (2, 1)          & (2, 6)          & (3, 1)          & (3, 6)          & (6, 0)          \\
            \hline
            (2, 1)          & (2, 1)          & (3, 6)          & \(\mathcal{O}\) & (2, 6)          & (6, 0)          & (3, 1)          \\
            \hline
            (2, 6)          & (2, 6)          & \(\mathcal{O}\) & (3, 1)          & (6, 0)          & (2, 1)          & (3, 6)          \\
            \hline
            (3, 1)          & (3, 1)          & (2, 6)          & (6, 0)          & (3, 6)          & \(\mathcal{O}\) & (2, 1)          \\
            \hline
            (3, 6)          & (3, 6)          & (6, 0)          & (2, 1)          & \(\mathcal{O}\) & (3, 1)          & (2, 6)          \\
            \hline
            (6, 0)          & (6, 0)          & (3, 1)          & (3, 6)          & (2, 1)          & (2, 6)          & \(\mathcal{O}\) \\
            \hline
          \end{tabular}
        \end{table}
      \end{solution}
    \end{parts}
  \end{questions}
\end{document}
