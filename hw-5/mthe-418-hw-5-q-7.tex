% region Filename parsing.
% Provides macros manipulating strings of tokens.
\RequirePackage{xstring}

% Store the jobname as a string with category 11 characters.
\edef\normaljobname{\expandafter\scantokens\expandafter{\jobname\noexpand}}
\StrBetween{\normaljobname}{hw-}{-q}[\homeworknumber]
\StrBehind{\normaljobname}{-q-}[\questionnumber]
% endregion Filename parsing.

\documentclass[
  coursecode={MTHE 418},
  assignmentname={Homework \homeworknumber},
  studentnumber=20053722,
  name={Bryan Hoang},
  draft,
  % final,
]{
  ltxanswer%
}

\usepackage{bch-style}

\begin{document}
  \begin{questions}
    \setcounter{question}{\questionnumber}
    \addtocounter{question}{-1}
    \question[10]\
    \begin{parts}
      \part{}
      \begin{solution}
        \begin{table}
          \caption{Computing \(19 \cdot (24, 14)\) on \(Y^{2} = X^{3} + 23X + 13 \Mod{83}\).}
          \begin{tabular}{*{4}{c}}
            \toprule
            Step \(i\) & \(n\) & \(Q = 2^{i} P\) & \(R\)           \\
            \midrule
            0          & 19    & (24, 14)        & \(\mathcal{O}\) \\
            1          & 9     & (30, 8)         & (24, 14)        \\
            2          & 4     & (24, 69)        & (30, 75)        \\
            3          & 2     & (30, 75)        & (30, 75)        \\
            4          & 1     & (24, 14)        & (30, 75)        \\
            5          & 0     & (30, 8)         & (24, 69)        \\
            \bottomrule
          \end{tabular}
        \end{table}
        Therefore, \(\boxed{19 \cdot (24, 14) = (24, 69)}\).
      \end{solution}
    \end{parts}
  \end{questions}
\end{document}
