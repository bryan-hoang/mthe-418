% region Filename parsing.
% Provides macros manipulating strings of tokens.
\RequirePackage{xstring}

% Store the jobname as a string with category 11 characters.
\edef\normaljobname{\expandafter\scantokens\expandafter{\jobname\noexpand}}
\StrBetween{\normaljobname}{hw-}{-q}[\homeworknumber]
\StrBehind{\normaljobname}{-q-}[\questionnumber]
% endregion Filename parsing.

\documentclass[
  coursecode={MTHE 418},
  assignmentname={Homework \homeworknumber},
  studentnumber=20053722,
  name={Bryan Hoang},
  draft,
  % final,
]{
  ltxanswer%
}

\usepackage{bch-style}

\begin{document}
  \begin{questions}
    \setcounter{question}{\questionnumber}
    \addtocounter{question}{-1}
    \question[10]\
    \begin{parts}
      \part{}
      \begin{solution}
        Bob should send the point \(Q_{B} = n_{B} P = 1943 (1980, 431) = \boxed{(1432, 667)} \in E(\mathbb{F}_{2671})\) to Alice.
      \end{solution}

      \part{}
      \begin{solution}
        Their shared secret value is the \(x\)-coordinate of the point \(n_{B} Q_{A} = 1943 (2110, 543) = (\boxed{2424}, 911) \in E(\mathbb{F}_{2671})\).
      \end{solution}

      \part{}
      \begin{solution}
        It seems pretty difficult for Eve to figure out Alice's secret multiplier \(n_{A}\) by hand. Through programming though, Eve can deduce that \(\boxed{n_{A} = 726}\).
      \end{solution}
    \end{parts}
  \end{questions}
\end{document}
