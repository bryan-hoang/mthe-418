% region Filename parsing.
% Provides macros manipulating strings of tokens.
\RequirePackage{xstring}

% Store the jobname as a string with category 11 characters.
\edef\normaljobname{\expandafter\scantokens\expandafter{\jobname\noexpand}}
\StrBetween{\normaljobname}{hw-}{-q}[\homeworknumber]
\StrBehind{\normaljobname}{-q-}[\questionnumber]
% endregion Filename parsing.

\documentclass[
  coursecode={MTHE 418},
  assignmentname={Homework \homeworknumber},
  studentnumber=20053722,
  name={Bryan Hoang},
  draft,
  % final,
]{
  ltxanswer%
}

\usepackage{bch-style}

\begin{document}
  \begin{questions}
    \setcounter{question}{\questionnumber}
    \addtocounter{question}{-1}
    \question[10]
    \begin{solution}
      \begin{answerfigure}
        \begin{tikzpicture}[x=1cm, y=1cm, z=-0.6cm]
          % Axes
          \draw [->] (0,0,0) -- (5,0,0) node [right] {$x$};
          \draw [->] (0,0,0) -- (0,5,0) node [left] {$y$};
          \draw [->] (0,0,0) -- (0,0,5) node [left] {$z$};
          % Vectors
          \draw [->, thick] (0,0,0) -- (1,3,-2);
          \draw [->, thick] (0,0,0) -- (2,1,0);
          \draw [->, thick] (0,0,0) -- (-1,2,5);
          % Labels
          \node [right] at (1,3,-2) {$\begin{bmatrix}1\\3\\-2\end{bmatrix}$};
          \node [right] at (2,1,0) {$\begin{bmatrix}2\\1\\0\end{bmatrix}$};
          \node [above] at (-1,2,5) {$\begin{bmatrix}-1\\2\\5\end{bmatrix}$};
        \end{tikzpicture}
        \caption{Plot of the vectors that form the fundamental domain.}
      \end{answerfigure}
      The volume of the fundamental domain formed by the basis vectors, \(V_{L}\), can be found by computing the determinant of the matrix formed by the vectors. Therefore, the volume of the fundamental domain is
      \begin{align*}
        V_{L} &= \abs*{\det\begin{pmatrix}
                               1  & 3 & -2 \\
                               2  & 1 & 0  \\
                               -1 & 2 & 5
                             \end{pmatrix}} \\
              &= \boxed{35}.
      \end{align*}
    \end{solution}
  \end{questions}
\end{document}
