% region Filename parsing.
% Provides macros manipulating strings of tokens.
\RequirePackage{xstring}

% Store the jobname as a string with category 11 characters.
\edef\normaljobname{\expandafter\scantokens\expandafter{\jobname\noexpand}}
\StrBetween{\normaljobname}{hw-}{-q}[\homeworknumber]
\StrBehind{\normaljobname}{-q-}[\questionnumber]
% endregion Filename parsing.

\documentclass[
  coursecode={MTHE 418},
  assignmentname={Homework \homeworknumber},
  studentnumber=20053722,
  name={Bryan Hoang},
  draft,
  % final,
]{
  ltxanswer%
}

\usepackage{bch-style}

\begin{document}
  \begin{questions}
    \setcounter{question}{\questionnumber}
    \addtocounter{question}{-1}
    \question[10]\
    \begin{parts}
      \part{}
      \begin{solution}
        \begin{align*}
          \bm{w}                                         &= \underbrace{6.22}_{t_{1}} \bm{v}_{1} + \underbrace{133.98}_{t_{2}} \bm{v}_{2} \\
          \Rightarrow                                    &\begin{cases}
                                                            a_{1} = \nint{t_{1}} = 6 \\
                                                            a_{2} = \nint{t_{2}} = 134
                                                          \end{cases}                               \\
          \Rightarrow \bm{v}                             &= 6 \bm{v}_{1} + 134 \bm{v}_{2}                                                 \\
                                                         &= \boxed{(43086, 11448)}                                                        \\
          \Rightarrow \alignedbox{\norm{\bm{v} - \bm{w}} &= 107.15}.
        \end{align*}
      \end{solution}

      \part{}
      \begin{solution}
        \begin{align*}
          \sqrt{\frac{\det(L)}{\norm{\bm{v}_{1}} \norm{\bm{v}_{2}}}} &\approx \sqrt{\frac{158709}{486.2 \cdot 329.2}} \\
          \alignedbox{                                               &\approx 0.9958}                                 \\
                                                                     &\approx 1.
        \end{align*}
        Since the ratio is close to 1, we can conclude that the basis \(\{\bm{v}_{1}, \bm{v}_{2}\}\) is a ``good'' basis.
      \end{solution}

      \part{}
      \begin{solution}
        \begin{proof}
          \begin{align*}
                                            &\boxed{\begin{cases}
                                                        \bm{v}_{1}^{\prime} = 5 \bm{v}_{1} + 6 \bm{v}_{2} \\
                                                        \bm{v}_{2}^{\prime} = 19 \bm{v}_{1} + 23 \bm{v}_{2}
                                                      \end{cases} } \\
            \Rightarrow \det\begin{pmatrix}
                              5  & 6  \\
                              19 & 23
                            \end{pmatrix} &= \boxed{1}.
          \end{align*}
        \end{proof}
      \end{solution}

      \part{}
      \begin{solution}
        \begin{align*}
          \bm{w}                                                  &= \underbrace{-2402.52}_{t_{1}} \bm{v}_{1}^{\prime} + \underbrace{632.57}_{t_{2}} \bm{v}_{2}^{\prime} \\
          \Rightarrow                                             &\begin{cases}
                                                                     a_{1} = \nint{t_{1}} = -2403 \\
                                                                     a_{2} = \nint{t_{2}} = 633
                                                                   \end{cases}                                                  \\
          \Rightarrow \bm{v}^{\prime}                             &= -2403 \bm{v}_{1}^{\prime} + 633 \bm{v}_{2}^{\prime}                                                 \\
                                                                  &= \boxed{(46548, 9561)}                                                                               \\
          \Rightarrow \alignedbox{\norm{\bm{v}^{\prime} - \bm{w}} &= 3860.08},                                                                                           \\
                                                                  &\gg 107.15                                                                                            \\
                                                                  &= \norm{\bm{v} - \bm{w}}
        \end{align*}
      \end{solution}

      \part{}
      \begin{solution}
        \begin{align*}
          \sqrt{\frac{\det(L)}{\norm{\bm{v}_{1}} \norm{\bm{v}_{2}}}} &\approx \sqrt{\frac{158709}{3323.2 \cdot 12673.8}} \\
          \alignedbox{                                               &\approx 0.061}                                     \\
                                                                     &\ll 1.
        \end{align*}
        Since the ratio much smaller than 1, we can conclude that the basis \(\{\bm{v}_{1}, \bm{v}_{2}\}\) is a ``bad'' basis.
      \end{solution}
    \end{parts}
  \end{questions}
\end{document}
